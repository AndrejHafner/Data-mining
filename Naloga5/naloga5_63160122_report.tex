% To je predloga za poročila o domačih nalogah pri predmetih, katerih
% nosilec je Blaž Zupan. Seveda lahko tudi dodaš kakšen nov, zanimiv
% in uporaben element, ki ga v tej predlogi (še) ni. Več o LaTeX-u izveš na
% spletu, na primer na http://tobi.oetiker.ch/lshort/lshort.pdf.
%
% To predlogo lahko spremeniš v PDF dokument s pomočjo programa
% pdflatex, ki je del standardne instalacije LaTeX programov.

\documentclass[a4paper,11pt]{article}
\usepackage{a4wide}
\usepackage{fullpage}
\usepackage[utf8x]{inputenc}
\usepackage{float}
\usepackage[slovene]{babel}
\selectlanguage{slovene}
\usepackage[toc,page]{appendix}
\usepackage[pdftex]{graphicx} % za slike
\usepackage{setspace}
\usepackage{color}
\definecolor{light-gray}{gray}{0.95}
\usepackage{listings} % za vključevanje kode
\usepackage{hyperref}
\usepackage{subfig}
\usepackage{titlesec}
\usepackage{flafter}
\renewcommand{\baselinestretch}{1.2} % za boljšo berljivost večji razmak
\renewcommand{\appendixpagename}{\normalfont\Large\bfseries{Priloge}}


\titleformat{name=\section}[runin]
  {\normalfont\bfseries}{}{0em}{}
\titleformat{name=\subsection}[runin]
  {\normalfont\bfseries}{}{0em}{}


% header
\makeatletter
\def\@maketitle{%
  \noindent
  \begin{minipage}{2in}
  \@author
  \end{minipage}
  \hfill
  \begin{minipage}{1.2in}
  \textbf{\@title}
  \end{minipage}
  \hfill
  \begin{minipage}{1.2in}
  \@date
  \end{minipage}
  \par
  \vskip 1.5em}
\makeatother


\lstset{ % nastavitve za izpis kode, sem lahko tudi kaj dodaš/spremeniš
language=Python,
basicstyle=\footnotesize,
basicstyle=\ttfamily\footnotesize\setstretch{1},
backgroundcolor=\color{light-gray},
}


% Naloga
\title{Naloga 5}
% Ime Priimek (vpisna)
\author{Andrej Hafner (63160122)}
\date{\today}

\begin{document}

\maketitle


\normalfont\textsl{}
\section{Opis metod.}

\begin{description}
\item[Matrična faktorizacija brez pristranosti] Implementirana je bila matrična faktorizacija, brez pristranosti uporabnika ali izvajalca. Iteracije se izvajajo dokler je RMSE na validacijski množici manjši kot v prejšnji iteraciji. 

\item[Matrična faktorizacija z pristranostjo] Implementirana je bila enaka matrična faktorizacija kot v prvi metodi, z dodano pristranostjo uporabnika in izvajalca. To je bilo dodano kot enice v predzadnjem stolpcu v matriki P in kot enice v zadnji vrstici matrike Q, ter z majhno spremembo v gradientnem spustu. 
\item[Matrična faktorizacija z pristranostjo in oznakami izvajalcev] Enako kot prejšnja metoda, sprememba nastane pri napovedovanju ocene izvajalca. Opazil sem, da je v testni množici prisotnih 1100 izvajalcev, ki jih ni v učni. Uporabil sem podatke iz \textit{user\_taggedartist.dat} da sem pridobil največkrat označeno oznako za vsakega izvajalca. V primeru, da je bilo potrebno napovedati oceno za izvajalca, ki ga ni bilo v učni množici, je bila ocena povprečje ocen uporabnika za izvajalce z enako največkrat označeno oznako.
\end{description}

\section{Rezultati.} 

\begin{table}[!htb]
	\centering
		\begin{tabular}{llp{4.3cm}}
			\hline
			 Ime metode & RMSE (učni podatki) & RMSE (tekmovalni strežnik) \\
			\hline
			Matrična faktorizacija brez pristranosti & 2.143  & 2.267 \\
			Matrična faktorizacija z pristranostjo & 1.716  & 1.810 \\
			Matrična faktorizacija z pristranostjo  & 1.716  & 1.797 \\
			in oznakami izvajalcev & & \\
			\hline
		\end{tabular}
\end{table}





\section{Izjava o izdelavi domače naloge.}
Domačo nalogo in pripadajoče programe sem izdelal sam.


\end{document}
